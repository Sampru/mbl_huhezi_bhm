\chapter{Etorkizunerako ildoak}\label{cha:etorkizuna}

Txosten honetan diseinatu eta definitzen den irakasleentzako formakuntzak eta orokorrean landu den Master Bukaerako Lan honek, etorkizunera begira hainbat lan lerro ireki eta garatzeko aukera eskaintzen du, gaur gaurkoz denbora mugatu izan baitu lan hau burutzeko. Etorkizunean garatzeko aukera duten lan ildoei dagokienez, lehenik eta behin, formakuntza saio hau ikastetxean gauzatzen den momentuan eta irakasleek bukaeran betetzen duten ebaluazio taula kontuan edukiz, hainbat hobekuntza aplikatu beharko lirateke. Etengabeko hobekuntza ardatz edukiz, beste ikuspegi batzuk kontuan hartuz, geroz eta formakuntza eraginkorrago bat lortzeko.

Horrez gain, formakuntza saio honen bitartez lortu nahi dena hezkuntzan erabiltzen diren baliabide digitalen erabilera kritiko bat egitea eta aldi berean, irakaslegoa baliabide digital libreetan trebatzea, ondoren ikasgelan erabiltzeko. Horregatik, formakuntza saioan eskuratutako jakintza horiek guztiak aurrera begira ikastetxe guztira zabaltzea ezinbestekoa da. Alde batetik, irakasleei baina baita ere guraso, zuzendaritza eta ikasleei, hau da, ikastetxeko hezkuntza komunitate osoari.
Bertatik, lan- talde bat sortu beharko litzateke prozesu honen gidaritza hartuko lukeena eta ez soilik pertsona batengan erortzea ardura guztia, nahiz eta ikastetxeko partaideen ardura izan arlo honen eraldaketa.

Modu horretara, komunitate osoa inguratzen dituen errealitatearen kontzientzia hartuta, hurrengo pausua ikastetxeko diagnosi sakon bat egitea izango litzateke, formakuntza saioan egingo diren behaketa taulak eta profil ezberdinetako iritziak kontuan hartuta. Behin, diagnosia oinarri hartuta, egungo errealitatea irauli eta burujabetza digitalerako bidean, bide- orri bat diseinatu beharko litzateke hurrengo pausuak definitzeko. Horretarako, garrantzitsua izango litzateke baliabide digital libreak sortzen dituzten enpresekin kontaktuan jartzea, proposamen pertsonalizatuak diseinatzeko ondoren ikastetxean txertatzeko eta horien erabilera egoki bat egiteko formakuntza eskaintzeko.

Honetaz guztiaz gain, beharrezkoa ikusten da irakasleen konpetentzia digitalen maila aztertzeaz gain (DigCompEdu markoan oinarrituz), hauen burujabetza digital maila analizatuko duen azterketa bat gauzatzea, bertatik informazio oso baliagarria eskuratuko delako egungo errealitatea ulertzeko eta etorkizunean martxan jarri beharreko proiektuak definitzeko.
