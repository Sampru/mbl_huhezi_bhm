%%%%%%%%%%%%%
%% SARRERA %%
%%%%%%%%%%%%%
\chapter{Sarrera}\label{cha:sarrera}
Bigarren Hezkuntzako Master amaierako lan hau irakasleen formakuntza saio bat diseinatzean datza, Praktika Komunitate Profesionalaren markoaren baitan. Gaur egun bizi garen gizartea, etengabeko aldaketek eta garapen zientifiko eta teknologikoak ezaugarritzen dute. Testuinguru honetan, nahiz eta, garapen honek hobekuntzak ekarri egunerokotasuneko bizitzetara, herrialde aberatsen eta pobreen arteko arrakala asko handitu du adibidez, eta baliabide teknologiko hauen parte handi bat multinazional gutxi batzuen esku dago.

Egoera honek zuzenean eragiten du hezkuntza sisteman ere, bertan zabalduena dauden baliabide teknologikoak GAFAM (\ref{sec:gafam}. atalean azalduko dena) enpresen esku baitaude. Modu horretan, alde batik, enpresek hezkuntza sisteman parte hartzen dute eta horrek datu kantitate handietara sarbidea bermatzen die, ondoren datu horiekin soilik multinazional handientzako mesedegarriak izango diren negozio berriak sortzeko. Beste alde batetik, ikasle eta orohar erabiltzaileen pribatutasuna eta norbere datuen kontrola erabat galtzen da.

Honen aurrean, beharrezkoa da inguratzen gaituen errealitatearekiko kontzientzia hartu eta ikuspegi kritiko batetik aztertzea egoera, ondoren egoera honi aurre egingo dioten planteamenduak garatzeko. Horretarako, hezkuntzaren eremua sektore klabe moduan identifikatzen da, hein handi batean aldaketak gauzatzeko baldintzak sortzeko gaitasuna dagoelako bertatik, egungo ikasleei beste balio batzuetan hezi eta haziz.

Baina hau guztia aurrera eramateko ezinbestekoa da irakaslegoak honekiko prestakuntza eduki eta kontziente izatea. Hori dela eta, irakasleentzako diseinatu den formakuntza saio honek hiru alderdi nagusi lantzea du helburu: GAFAM enpresek hezkuntzan aurrera eramaten duten esku- hartzea ezaugarritzea, egoera honi aurre egin eta burujabetza digitalera bidean martxan dauden proiektuen oinarriak azaldu eta beraien planteamenduak ezagutzera ematea eta baliabide teknologiko libreak ezagutarazi, hauek aurkitzen jakiteko gaitasunak eskuratu eta hauen erabileran irakasleak trebatzea.

Dokumentuaren egiturari dagokionez, hasteko eta behin, lanaren helburuak definitu dira eta proiektuaren justifikazioa egin da. Jarraian, irakasleen formakuntzak izango dituen oinarri teorikoak garatu dira, marko teorikoaren atalaren barruan. Ondoren, Praktika Komunitate Profesionalaren markoan oinarritzen den formakuntza saioa aurkezten da.

Azkenik, garatutako formakuntza saiotik eta oinarri teorikoetatik ateratako hainbat ondorio aurkezten dira eta etorkizunean garatu daitezkeen lan lerro posibleak definitu dira.\\

%%%%%%%%%%%%%%%
%% HELBURUAK %%
%%%%%%%%%%%%%%%
{\let\clearpage\relax \chapter{Helburuak}\label{cha:helburuak}}

Bi helburu mota bereizi daitezke. Lehenik eta behin, master bukaerako lan honen helburu orokorrak finkatuko dira. Ondoren, eta garatu den formakuntza saioaren helburuak finkatuko dira.

Hasteko, eta txostenaren helburuak begiratuz, hainbat puntu azpimarra daitezke:
\begin{itemize}
    \item DigCompEdu (\ref{subsec:digcompedu}. atalean sakonduko dena) markoan oinarrituta EAEko ikastetxeetako digitalizazio maila aztertzea.
    \item Hezkuntzaren sektorean interesa erakutsi duten enpresen interes eta helburuak kritikoki aztertzea.
    \item Burujabetza digitalaren inguruko hausnarketa egitea eta kontzientzia garatzea.
    \item Baliabide libreak identifikatzea eta horietan trebatzea.
    \item Irakasleentzako formakuntza saio bat diseinatzea.
\end{itemize}

Honekin jarraituz, MBLaren helburu nagusia irakasleentzako formakuntza saio bat burutzea da. Goian dauden helburuak irakasle komunitatera hedatzeko beharra dagoela identifikatu da, eta formakuntza saioez baliatuz egin behar dela uste da. Beraz, saio honek bete behar dituen helburuak honako hauek dira:
\begin{itemize}
    \item Derrigorrezko Bigarren Hezkuntzako (DBH) eta Batxilergoko irakasleentzat diseinatzea.
    \item Metodologia aktiboak erabiltzea, kurtsoaren eraginkortasuna areagotzeko.
    \item PKP markoan oinarritutako irakasle lantaldea eratzea.
    \item Baliabide digitalak erabiltzea, ahal den heinean, multinazional handien \textit{software}tik\footnote{Programa informatiko.} kanpo.
    \item Irakasle bakoitzak bere buruaren progresuaren gaineko hausnarketa egitea, autoebaluazioaren bidez.
    \item Ikastetxean aldaketak bultzatzeko, eskuratutako jakintzak komunitate guztira zabaltzea.
    \item Etengabeko hobekuntza ardatz, irakasleek jasotako formakuntza saioaren gaineko ebaluazioa egitea.
\end{itemize}

Helburu hauek bete diren dokumentuaren amaieran ondorioztatuko da, \ref{cha:ondorioak}. atalean.\\

%%%%%%%%%%%%%%%%%%%
%% JUSTIFIKAZIOA %%
%%%%%%%%%%%%%%%%%%%
{\let\clearpage\relax \chapter{Justifikazioa}\label{cha:justifikazioa}}
Gizarte mailan, transformazio teknologiko eta digitalak presentzia handia hartu du azken urteotan eta honek bizitzako esparru desberdinetan gaur egun arte izan duen eragina kontuan hartuta, inpaktu handia izango du etorkizun laburrean.

Izan ere, eskola ez da ekosistema itxi bat, baizik gizarte eta ekosistema komun baten barruan dago: horregatik, zergatik eta zertarako hezkuntza teknologiko-zientifiko-matematikoari galdera erantzunez, hainbat erronka azaleratzen dira gizarte mailan. Hauen artean, aipatutako transformazio teknologiko eta digitalak, hezkuntza teknologikoan eragin zuzena duen problematika dago. Teknologiaren garapen ikaragarria, oso denbora laburrean ematen ari da eta hezkuntza sisteman aldaketak nabarmenak dira. Multinazional handiek teknologia digitalen merkatua monopolizatuta dute eta hezkuntza sistemaren parte dira jada. Horregatik, hezkuntza sisteman dugun erronka nagusienetako bat burujabetza digitalerako pausuak ematea da.

\citeauthor{ehd2021manifestua}-k (\citeyear{ehd2021manifestua}), XX. mende amaieratik hezkuntza sakoneko eraldaketa batean murgildurik dagoela dio. Digitalizazioa eskoletara iristeak prozesu hau areagotu du eredu zaharraren pilareak kolpatzen dituen heinean. baina erantzuten dituenak baino galdera gehiago sortzen ditu, Pandemiarekin hezkuntzaren beharrak aztertuz baina, galdera gehiago sortzen ditu: presentzialtasunaren garrantzia, ingurune digitalaren jabea nor izan behar den, nola bermatzen diren hezkuntzarako eskubidea eta abar. Digitalizazioak erronka zaharrei erantzuteko galdera berriak planteatzen ditu hezkuntzan ere.

Azken finean, helburua, teknologia eta ikuspuntu teknologikoa gizartearen zerbitzura egongo dena, pertsonen bizitzak hobetzeko erabiltzea da. Teknologia ez baita neutrala, teknologia guk (gizarteak) egiten dugu eta hezkuntza enpresa multinazionalen eskuetan utzi dugun honetan, hauek hezkuntzara hurbiltzeaz gain, haien ekosistema digitalak ikastetxeetan infiltratu dira. 

Honi mugak jarri eta burujabetza digitalaren bidean ikastetxeek aurrera pausuak emateko, beharrezkoa ikusten dugu irakasleak, herritar ororen moduan, konpetentzia teknologiko-digitaletan oinarrizko ezagutzak izatea eta hauetan eguneratuta egotea. Trebezietatik haratago, ikuspegi kritiko bat garatu eta digitalizazioaren nolakotasunean fokua jartzea, “etikoa, arduratsua, eraginkorra eta euskalduna” hain zuzen ere.

Gauzak horrela, hezkuntzaren gaur egungo eta etorkizuneko erronka nagusiei erantzuteko irakasleak prestatu behar ditugu. Horregatik, Master Bukaerako Lan honetan PKP markoan oinarritutako formakuntza saio baten diseinua egitea erabaki da.
