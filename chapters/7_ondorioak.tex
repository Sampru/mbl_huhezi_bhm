\chapter{Ondorioak}\label{cha:ondorioak}

Txosten honetan, \ref{cha:helburuak}. atalean, finkatu diren helburuak aztertuz, hainbat ondorio atera daitezke. Aldez aurretik, formakuntza saioaren eraginkortasuna aztertzeko testuinguru erreal batean kokatzea (ikastetxe batean martxan jartzea) falta dela aipatu nahi da. Beraz, ondorio hauek zorrozteko, formakuntza saioaren ebaluaketa bat faltako zen.

Lehenik eta behin,  digitalizazio mailaren azterketaren inguruan,  helburua bete dela esan daiteke. DigCompEdu markoa aztertuz, eta EAEko testuinguru digitalaren irudi errealista bat lortu da, ondoren hezkuntzan esku hartzen duten enpresekin erlazionatzeko. Baina, nahiz eta DigCompEdu markoa erabilgarria izan, ez du burujabetza digitalean inolako erreparorik egiten, konpetentzia digitalak bakarrik neurtzen ditu. Hau honela izanda, etorkizun baterako, eta formakuntza saioa hobetzeko asmoz, bestelako ikerketa bat egin beharko litzateke, EAEko irakaslegoaren enpresa handiekiko dependentzia maila aztertzen duena.

Burujabetza mailarekin jarraituz, eta aurrekarietan fokua jarrita, EAEko, Espainiar estatuko eta  Europar mailako burujabetza maila zein den ikusi da, GAFAM enpresek duten eragina aztertuta. Honetaz gain, mugimendu desberdinak aztertu dira, eta hauek duten oinarriak ikertu eta formakuntzan parte hartu dute. Arlo honi dagokionez, formakuntza saioaren helburu nagusia izanda, helburua bete dela esan daiteke.

Amaitzeko, formakuntza saioaren ondorioak bilduko dira. Lehenengo ondorioa honakoa da: ez da DBHko eta batxilergoko irakasleentzat bideratutako formakuntza saioa egin, jarraian azalduko da arrazoia. Formakuntzaren edukiak direla eta,  ezin daiteke esan bi etapa hauetako irakasleentzako formakuntza denik espresuki. Ikastetxeetako burujabetza digitala sustatzean zentratua dagoen saioa denez, etapa hauetatik at ere balio dezake. Baina egia da ere konpetentzia digitalaren garapena, EAEko hezkuntza curriculumean biltzen den bezala, etapa hauetan burutzen dela. Beraz, helburu hau bete den ala ez kontutan izateko orduan, baietz esan daiteke.

Metodologia aktiboei dagokionez, formakuntza saioan gauzatutako dinamiken bitartez metodologia aktiboak erabiltzea sustatu da. Alde batetik, ekipo lanean aritzea bermatu da uneoro, bai talde txiki zein irakasle lantaldeko komunitate moduan. Horretarako, irakaslearen aurre ezagutzak piztu dira saio bakoitzaren hasieran, baliabide digital eta euskarri desberdinak erabiliz formakuntzaren helburu nagusia kontuan hartuta praktika eredugarria izatea bermatu da. Azkenik, bai irakaslearen progresua neurtzeko eta formakuntza saioaren feedback-a jasotzeko ebaluazioak gauzatu dira.

Azken helburu honekin lotuta, formakuntza saioa komunitatea eratzera zentratua izan da. Saioetan burututako ikasketa metodo dinamikoez gain, talde eboluzioan zentratu da, eta saioetatik kanpo ere ikastetxean begirada jarri eta etorkizuneko erronkak kontuan hartuta, ikastetxerako epe luzerako helburuak definitu dira eta hauek gauzatzeko plan bat. Horretarako, irakaslearen ikasketa prozesua bultzatu da, formakuntza saiotik kanpo, sakontzeko baliabideak eskainiz eta irakasle lantaldean jakintzak partekatuz.

Hau dela eta, formakuntza saioa metodologia aktiboetan oinarritua izan dela esan daiteke, eta praktika komunitate bat eratzeko bidea egin duela.

Kurtsoa burujabetza digitalaren inguruko hausnarketa bezala ulertuta, eta baliabide digital libreen erabileraren aldeko testua izanik, hipokresia izango litzateke baliabide ireki hauen erabilerarik ez egitea. Kurtsoa osatzen duen materialaren ehuneko handi bat baliabide irekiak erabiliz sortu da, baina ezin izan da osoki baliabide digital askeetan oinarritu. Dokumentu honetan zehar azaltzen den bezala, norbanakoaren lanak ez dauka eragin handirik, eta komunitate baten laguntza behar da burujabetza hau eskuratzeko. MBL hau, hezkuntza sistemaren parte izanik, ez da bere kabuz baliabide digital libreetan oinarritzeko gai. Helburua ez da bete, baina bi gauza argi geratu dira horrela: multinazionalen menpekotasuna existitzen dela, eta burujabetza digitala talde lanarekin (bai irakasle, bai ikasle eta bai eskolako azpiegitura osatzen duten langile guztien  lanarekin) lortu behar den gauza bat dela.

Bukatzeko, ebaluazioaren inguruko helburuen inguruan, eta berriz ere, praktikan jarri arte ondorio erreal bat ateratzeko jakintzarik izan gabe, bai irakasleen autoebaluaziorako eta bai kurtsoaren feedbackerako tresnak sortu dira. Honekin, irakasleek beraien burua ebaluatu eta formakuntzaren ebaluazio bat egitea espero da, baina ezin daiteke ondorio gehiagorik atera.  
