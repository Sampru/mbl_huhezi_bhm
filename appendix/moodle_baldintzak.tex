\section{Moodle plataformaren erabilera baldintzak}\label{eranskin:moodle}
Moodle plataforma libre bat da, eta norberak zerbitzari pribatu batean montatu behar du. Bestalde, interneteko domeinuak ere ordainpeko plan batera daude lotuak.

Hau horrela izanda, “burujabetzadigitala.eus” domeinua Mondragon Unibertsitateak PuntuEUS fundazioarekin duen eskaintza erabiliz eskuratu da, non 364 egunerako domeinu bat doan erregistratu daitekeen, ikasle izanez gero. Hori dela eta, ez da bermatzen 2023ko ekainaren 11tik aurrera sarbidea izaterik \url{http://burujabetzadigitala.eus/moodle} linka erabiliz. Zerbitzari hau islatuta dago “sampru.ovh” domeinuan, beraz, behin aurreko data igaro eta gero, posible da Moodle plataforma \url{http://sampru.ovh/moodle} helbidean eskuragarri izatea.

Aldi berean, zerbitzaria erabilera pribaturako da. Beharra izanez gero, Moodle plataforma kendu egingo da beste proiekturen bat garatzeko. Honekin adierazi nahi dena da ez dagoela inolako bermerik dokumentu hau irakurtzen ari den unean euskarri digital hau erabili ahal izateko.

Dena den, eta tesiaren formakuntza saioaren inguruko materiala eskuratu nahi baduzu, gunearen segurtasun kopia bat eskuragarri egongo da ondorengo \href{https://drive.google.com/drive/folders/1S8kl20k8nW0w-bopBuNT5HyozOxtO4Ei?usp=sharing}{linkean}.